\subsection{Prodotto finale}
Al termine del progetto di stage, il software prodotto ha soddisfatto tutti i requisiti obbligatori individuati, con buona soddisfazione da parte del committente. \\ \\
I requisiti sulla generazione delle statistiche non sono stati realizzabili in quanto, data la contemporaneità dello sviluppo del sistema nel suo complesso, non si è riusciti ad avere un'idea precisa dei tempi necessari alla raccolta di dati sufficientemente dettagliati per la generazione delle statistiche individuate. Ciò è stato causato soprattutto dalla scarsa conoscenza del team con le tecniche di \textit{tracking}, che si sono invece rivelate più complicate di quanto aspettato. \\ \\
Nonostante tale lacuna il prodotto è attualmente in grado di svolgere il compito per cui è stato realizzato, infatti esso è attualmente utilizzato (ancora in fase di \textit{beta testing}) all'interno degli uffici di WCap per la generazione dei primi risultati di tracciamento.

\subsection{Conoscenze acquisite}
Durante l'esperienza ho approfondito le mie capacità in ognuna delle fasi principali dello sviluppo software. Le attività di verifica sono state trattate con particolare cura, anche tramite la ricerca e l'uso di programmi atti allo scopo. Ho inoltre avuto modo di sviluppare competenze in diverse tecnologie e nel loro uso.
Le più rilevanti dal mio punto di vista sono le seguenti:
\begin{itemize}
	\item \textbf{C++} essendo un linguaggio anziano e di basso livello è stato importante approfondire la conoscenza e l'esperienza nell'uso di C++. Infatti ritengo che sia fondamentale la pratica per raggiungere un buon livello nella stesura di codice pulito e manutenibile. Avendo già una buona base nell'uso del linguaggio non ho incontrato difficoltà nel suo utilizzo, ma ho appreso ulteriori standard e convenzioni di scrittura riconosciuti da tutta la comunità. Ho inoltre approfondito molti strumenti che forniscono un buon supporto all'uso di tale linguaggio, e che permettono quindi di monitorare i difetti e l'andamento della scrittura del codice. In particolare Valgrind e CppDepend si sono rivelati degli strumenti molto interessanti e utili ai fini dell'analisi del codice sorgente
	\item \textbf{ORM} ho avuto modo di conoscere ed apprendere le basi della tecnica di \textit{object relational mapping}, che prima di tale esperienza mi era sconosciuta. Ritengo che al giorno d'oggi l'utilizzo di una libreria che gestisca l'accesso ai dati in maniera astratta (e ad oggetti) sia fondamentale in ogni applicazione, e permetta anche di risparmiare al programmatore l'elevato tempo richiesto dalla stesura di tutte le parti di \textit{data management}
	\item \textbf{CMake} trovo che sia stato molto interessante l'apprendimento del sistema di build di CMake, in quanto ho notato che esso è uno dei più utilizzati dai progetti attuali per le sue caratteristiche di portabilità. Mi ha permesso inoltre di acquisire una certa dinamicità nell'utilizzo di un sistema di build, oltre che avermi fatto conoscere brevemente le tecniche di stesura di un \textit{makefile}. Ritengo che un sistema di build \textit{cross-platform} che permetta di astrarre dalle specifiche della piattaforma sia fondamentale per qualsiasi progetto attuale
	\item \textbf{Computer Vision} nonostante nel progetto ci si sia scontrati con problematiche relativamente semplici riguardanti la configurazione delle telecamere, ho avuto modo di conoscere e approfondire diversi aspetti nell'ambito della \textit{computer vision}. Infatti ho avuto modo di dedicarmi anche allo studio del sistema di \textit{video analytics} (realizzato da un altro membro del team di Pathflow). Era necessario che io avessi una visione delle tecniche, degli algoritmi utilizzati e dell'architettura del sistema realizzato, in quanto vi era la possibilità che la responsabilità di tale parte del sistema mi fosse affidata. 
	Mi è stato necessario quindi imparare la teoria alla base di tecniche molto complesse, tra le quali:
	\begin{itemize}
		\item \textit{Background subtraction}
		\item \textit{Mean shift tracking} e \textit{Cam shift tracking}
		\item \textit{Kalman filtering}
		\item \textit{Occlusions}
	\end{itemize}
	Trovo che l'ambito della \textit{computer vision} sia molto interessante e stimolante, mi è dispiaciuto non poterla trattare con più dettaglio nel progetto di stage, riconosco però che essa avrebbe richiesto molto più tempo per la realizzazione di risultati concreti.
\end{itemize}

\subsection{Prospettive dell'applicazione}
La parte del sistema che è stata realizzata è (per i limiti attuali della tipologia dei dati raccolti) completa nelle sue funzionalità. Essa in futuro dovrà essere in grado di generare le statistiche richieste, nonché di comunicare con un server remoto per l'invio di dati. Tali dati dovranno essere aggregati, e verranno presentati al cliente tramite una piattaforma web di analytics, che presenterà sia statistiche numeriche che grafici. \\ \\
Molto impegno si sta dedicando allo studio di soluzioni al problema di \textit{occlusion} durante il tracking degli oggetti. Infatti in un contesto reale tale eventualità (l'\textit{occlusion}) è molto frequente, ed è quindi necessario studiare una soluzione o delle tecniche di gestione che permettano di non perdere il tracciamento.\\ \\
Il maggior limite attuale del sistema nel suo complesso è dato dalla scarsa raffinatezza dei dati raccolti. In futuro sarà necessario ricavare non solo le coordinate di tracciamento, ma anche informazioni più dettagliate. Particolare attenzione verrà posta nell'implementazione di una tecnica di identificazione univoca delle varie persone all'interno dell'area del locale. 



\subsection{Relazione con la preparazione accademica}
Nello svolgimento del progetto di stage ho avuto modo di applicare concretamente le conoscenze per la gestione di progetti software complessi introdotte dall'università. \\
E' stata fondamentale la capacità di suddividere il progetto in fasi distinte ognuna con le sue attività, essa mi è servita non solo per organizzare i lavoro, le tempistiche e le scadenze, gli obiettivi, ma anche per facilitare la comunicazione con i committenti. Infatti presentando i risultati delle varie attività durante l'evoluzione del progetto, sono riuscito a far avere una visione in \textit{real time} dei risultati che avevo raggiunto. \\ \\ 
Ritengo che i corsi accademici del percorso di laurea triennale siano stati fondamentali, in quanto essi mi hanno preparato ad risolvere al meglio le problematiche proposte in autonomia. Essi mi hanno permesso di apprendere una metodo con il quale affrontare i problemi molto dinamico e versatile. Con esso sono stato in grado di ricercare strumenti e confrontarli tra di loro con occhio critico, per effettuare scelte basate su elementi concreti, e che si adeguassero al meglio al tipo di problema. \\
Credo che questa apertura mentale sia fondamentale al giorno d'oggi per un \textit{software engineer}. Infatti, nel mondo della tecnologia, la mutevolezza degli strumenti e il loro elevato numero fanno sì che l'insegnamento di tecnologie specifiche per la soluzione a determinati problemi sia alle volte limitante (è auspicabile che dopo qualche tempo una soluzione migliore e più innovativa risolverà quella stessa problematica). Sono quindi molto soddisfatto del percorso di studi che ho affrontato per la dinamicità mentale che mi ha fornito. \\ \\
Il progetto di stage mi ha permesso di provare direttamente tutte le diverse attività di cui si compone lo sviluppo software. Sono molto contento delle ulteriori competenze che ho acquisito, e della strategia che ho utilizzato durante lo sviluppo. Ho avuto modo di provare in prima persona i vantaggi di una buona progettazione, che ha facilitato e velocizzato molto la codifica nonchè l'integrazione dei vari componenti. Inoltre ho acquisito ulteriore esperienza per quanto riguarda la verifica e la validazione del software, esplorando con cura alcune tecniche di analisi statica e dinamica molto interessanti. \\ \\ 

Un aspetto che avrei preferito venisse trattato con più cura è stato il lavoro di gruppo. Nel progetto di stage le responsabilità delle varie parti del sistema erano molto divise (e le relazioni tra di esse sono state accuratamente definite a priori), quindi è stato preferibile che ogni membro si occupasse di ogni parte in maniera autonoma. Questo non è necessariamente un punto negativo, ma dal mio punto di vista sarebbe stato più istruttivo (per quanto riguarda le competenze) un progetto che avesse implicato una stretta collaborazione con altri sviluppatori. \\ \\

Per concludere penso che lo stage sia stato un impegno molto formativo, che mi ha permesso di mettere in pratica le conoscenze e le capacità acquisite durante il percorso di studi. Mi ritengo soddisfatto del suo esito finale.