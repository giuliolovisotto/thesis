\subsubsection{Requisiti funzionali} \label{sec:reqfun}
\begin{center}
\begin{longtable}{ | l | p{5cm} | c | p{1.5cm} |}
\caption{Tabella requisiti funzionali} \\
\hline 
\textbf{Codice} & \textbf{Descrizione} & \textbf{Tipologia} & \textbf{Fonte} \\ \hline
\endfirsthead
\multicolumn{4}{c}%

{\tablename\ \thetable\ -- \textit{Continued from previous page}} \\
\hline
\textbf{Codice} & \textbf{Descrizione} & \textbf{Tipologia} & \textbf{Fonte} \\
\hline
\endhead
\hline \multicolumn{4}{r}{\textit{Continued on next page}} \\
\endfoot
\hline
\endlastfoot 
RF1 & Il sistema deve permettere all'utente di eseguire la calibrazione delle telecamere & Obbligatorio & UC1 \\ \hline
RF1.1 & Il sistema deve permettere all'utente di impostare le opzioni di calibrazione & Obbligatorio & UC1  UC1.1 \\ \hline
RF1.1.1 & Il sistema deve permettere all'utente di selezionare il numero di immagini da utilizzare per la calibrazione & Obbligatorio & UC1  UC1.1 UC1.1.1 \\ \hline
RF1.1.2 & Il sistema deve permettere all'utente di selezionare il frame step per la calibrazione & Obbligatorio & UC1  UC1.1 UC1.1.2 \\ \hline
RF1.1.3 & Il sistema deve permettere all'utente di selezionare l'indirizzo identificativo della telecamera & Obbligatorio & UC1.1 UC1.1.3 \\ \hline
RF1.1.4 & Il sistema deve eseguire la calibrazione delle telecamere con le opzioni impostate & Obbligatorio & UC1.1 UC1.1.4 \\ \hline
RF1.2 & Il sistema deve permettere all'utente di visualizzare un video che utilizza i parametri di calibrazione per  ottenere un video \textit{undistorted} & Desiderabile & UC1  UC1.2 \\ \hline
RF2 & Il sistema deve permettere all'utente di configurare le telecamere & Obbligatorio & UC2 \\ \hline
RF2.1 & Il sistema deve permettere all'utente di inserire una nuova telecamera & Obbligatorio & UC2 UC2.1 \\ \hline
RF2.2 & Il sistema deve permettere all'utente di rimuovere una telecamera & Obbligatorio & UC2 UC2.2 \\ \hline
RF2.3 & Il sistema deve permettere all'utente di modificare e salvare in maniera persistente la configurazione di una telecamera & Obbligatorio & UC2 UC2.3 \\ \hline
RF2.3.1 & Il sistema deve permettere all'utente di modificare il file di distortion di calibrazione di una telecamera & Desiderabile & UC2 UC2.3 UC2.3.1 \\ \hline
RF2.3.2 & Il sistema deve permettere all'utente di modificare il file di intrinsics di calibrazione di una telecamera & Desiderabile & UC2 UC2.3 UC2.3.2 \\ \hline
RF2.3.3 & Il sistema deve permettere all'utente di modificare il valore dell'altezza del frame utilizzato per una telecamera & Obbligatorio & UC2 UC2.3 UC2.3.3 \\ \hline
RF2.3.4 & Il sistema deve permettere all'utente di modificare il valore della larghezza del frame utilizzato per una telecamera & Obbligatorio & UC2 UC2.3 UC2.3.4 \\ \hline
RF2.3.5 & Il sistema deve permettere all'utente di modificare il file contenente la homography matrix utilizzata per la traduzione delle coordinate di tracking relative ad una telecamera & Obbligatorio & UC2 UC2.3 UC2.3.5 \\ \hline
RF2.4 & Il sistema deve permettere di salvare un frame preso dal video stream della telecamera & Obbligatorio & UC2 UC2.4 \\ \hline
RF2.5 & Il sistema deve permettere di convertire un file .DXF in un file .PNG che riproduce graficamente l'immagine contenuta nel file originale & Obbligatorio & Interno \\ \hline
RF2.6 & Il sistema deve permettere di calcolare la homography matrix utilizzata per la traduzione delle coordinate di tracking relative ad una telecamera & Obbligatorio & UC2 UC2.6 \\ \hline
RF2.7 & Il sistema deve permettere di selezionare una telecamera per la modifica & Obbligatorio & UC2 UC2.7 \\ \hline
RF3 & Il sistema deve permettere all'utente di generare delle statistiche a partire dai dati di tracking attualmente presenti & Obbligatorio & UC3 \\ \hline
RF3.1 & Il sistema deve permettere di trasformare i dati di tracking in coordinate di posizione all'interno di un immagine .PNG che descrive la planimetria del locale, e di salvare tali informazioni in maniera persistente & Obbligatorio & Interno \\ \hline
RF3.2 & Il sistema deve permettere di generare un heatmap grafica che fornisca una rappresentazione grafica dei dati di tracking & Obbligatorio & UC3 UC3.2 \\ \hline
RF3.3 & Il sistema deve permettere di generare delle statistiche numeriche a partire dai dati di tracking e di salvarle in maniera persistente & Facoltativo & UC3 UC3.3 \\ \hline
RF3.3.1 & Il sistema deve permettere di generare la statistica di \textit{Dwell Time} & Facoltativo & UC3 UC3.3 \\ \hline
RF3.3.2 & Il sistema deve permettere di generare la statistica di \textit{Counting} & Facoltativo & UC3 UC3.3 \\ \hline
RF3.3.3 & Il sistema deve permettere di generare la statistica di \textit{Waiting Line} & Facoltativo & UC3 UC3.3 \\ \hline
\end{longtable}
\end{center}

\subsubsection{Requisiti di vincolo} \label{sec:reqvin}
\begin{center}
    \begin{longtable}{ | l | p{5cm} | c | p{1.5cm} |}
    \caption{Tabella requisiti di vincolo} \\
    \hline 
    \textbf{Codice} & \textbf{Descrizione} & \textbf{Tipologia} & \textbf{Fonte} \\ \hline
\endfirsthead
\multicolumn{4}{c}% 

{\tablename\ \thetable\ -- \textit{Continued from previous page}} \\
\hline
\textbf{Codice} & \textbf{Descrizione} & \textbf{Tipologia} & \textbf{Fonte} \\
\hline
\endhead
\hline \multicolumn{4}{r}{\textit{Continued on next page}} \\
\endfoot
\hline
\endlastfoot
RV1 & Il sistema dev'essere strutturato secondo un'architettura 3-tier & Obbligatorio & Capitolato \\ \hline 
RV2 & Il sistema deve funzionare in ambiente Linux & Obbligatorio & Capitolato \\ \hline 
RV3 & Il sistema deve funzionare in ambiente Windows & Facoltativo & Capitolato \\ \hline 
RV4 & Il sistema deve funzionare in ambiente Mac OS X & Desiderabile & Capitolato \\ \hline 
RV5 & Il sistema utilizzerà come strumento di build CMake, verranno quindi forniti i relativi file & Obbligatorio & Capitolato \\ \hline 
\end{longtable}
\end{center}

\subsubsection{Requisiti di qualita} \label{sec:reqqua}
\begin{center}
    \begin{longtable}{ | l | p{5cm} | c | p{1.5cm} |}
    \caption{Tabella requisiti di qualita} \\
    \hline 
    \textbf{Codice} & \textbf{Descrizione} & \textbf{Tipologia} & \textbf{Fonte} \\ \hline
\endfirsthead
\multicolumn{4}{c}% 

{\tablename\ \thetable\ -- \textit{Continued from previous page}} \\
\hline
\textbf{Codice} & \textbf{Descrizione} & \textbf{Tipologia} & \textbf{Fonte} \\
\hline
\endhead
\hline \multicolumn{4}{r}{\textit{Continued on next page}} \\
\endfoot
\hline
\endlastfoot
RQ1 & Deve essere prodotta documentazione del codice sorgente del software & Obbligatorio & Interno \\ \hline 
RQ2 & Il sistema deve garantire la massima indipendenza tra le funzionalità & Desiderabile & Capitolato \\ \hline 
\end{longtable}
\end{center}

