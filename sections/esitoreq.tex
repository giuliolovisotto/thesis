\subsubsection{Requisiti di vincolo} \label{sec:reqvin} \begin{center}
\begin{longtable}{ | l | p{5cm} | c | c |}
\caption{Tabella di soddisfacimento dei requisiti funzionali} \\
\hline 
\textbf{Codice} & \textbf{Descrizione} & \textbf{Tipologia} & \textbf{Esito} \\ \hline
\endfirsthead
\multicolumn{4}{c}%

{\tablename\ \thetable\ -- \textit{Continued from previous page}} \\
\hline
\textbf{Codice} & \textbf{Descrizione} & \textbf{Tipologia} & \textbf{Esito} \\
\hline
\endhead
\hline \multicolumn{4}{r}{\textit{Continued on next page}} \\
\endfoot
\hline
\endlastfoot 
RF1 & Il sistema deve permettere all'utente di eseguire la calibrazione delle telecamere & Obbligatorio &  \textcolor{green!80!blue}{soddisfatto}  \\ \hline 
RF1.1 & Il sistema deve permettere all'utente di impostare le opzioni di calibrazione & Obbligatorio &  \textcolor{green!80!blue}{soddisfatto}  \\ \hline 
RF1.1.1 & Il sistema deve permettere all'utente di selezionare il numero di immagini da utilizzare per la calibrazione & Obbligatorio &  \textcolor{green!80!blue}{soddisfatto}  \\ \hline 
RF1.1.2 & Il sistema deve permettere all'utente di selezionare il frame step per la calibrazione & Obbligatorio &  \textcolor{green!80!blue}{soddisfatto}  \\ \hline 
RF1.1.3 & Il sistema deve permettere all'utente di selezionare l'indirizzo identificativo della telecamera & Obbligatorio &  \textcolor{green!80!blue}{soddisfatto}  \\ \hline 
RF1.1.4 & Il sistema deve eseguire la calibrazione delle telecamere con le opzioni impostate & Obbligatorio &  \textcolor{green!80!blue}{soddisfatto}  \\ \hline 
RF1.2 & Il sistema deve permettere all'utente di visualizzare un video che utilizza i parametri di calibrazione per  ottenere un video \textit{undistorted} & Desiderabile &  \textcolor{green!80!blue}{soddisfatto}  \\ \hline 
RF2 & Il sistema deve permettere all'utente di configurare le telecamere & Obbligatorio &  \textcolor{green!80!blue}{soddisfatto}  \\ \hline 
RF2.1 & Il sistema deve permettere all'utente di inserire una nuova telecamera & Obbligatorio &  \textcolor{green!80!blue}{soddisfatto}  \\ \hline 
RF2.2 & Il sistema deve permettere all'utente di rimuovere una telecamera & Obbligatorio &  \textcolor{green!80!blue}{soddisfatto}  \\ \hline 
RF2.3 & Il sistema deve permettere all'utente di modificare e salvare in maniera persistente la configurazione di una telecamera & Obbligatorio &  \textcolor{green!80!blue}{soddisfatto}  \\ \hline 
RF2.3.1 & Il sistema deve permettere all'utente di modificare il file di distortion di calibrazione di una telecamera & Desiderabile &  \textcolor{green!80!blue}{soddisfatto}  \\ \hline 
RF2.3.2 & Il sistema deve permettere all'utente di modificare il file di intrinsics di calibrazione di una telecamera & Desiderabile &  \textcolor{green!80!blue}{soddisfatto}  \\ \hline 
RF2.3.3 & Il sistema deve permettere all'utente di modificare il valore dell'altezza del frame utilizzato per una telecamera & Obbligatorio &  \textcolor{green!80!blue}{soddisfatto}  \\ \hline 
RF2.3.4 & Il sistema deve permettere all'utente di modificare il valore della larghezza del frame utilizzato per una telecamera & Obbligatorio &  \textcolor{green!80!blue}{soddisfatto}  \\ \hline 
RF2.3.5 & Il sistema deve permettere all'utente di modificare il file contenente la homography matrix utilizzata per la traduzione delle coordinate di tracking relative ad una telecamera & Obbligatorio &  \textcolor{green!80!blue}{soddisfatto}  \\ \hline 
RF2.4 & Il sistema deve permettere di salvare un frame preso dal video stream della telecamera & Obbligatorio &  \textcolor{green!80!blue}{soddisfatto}  \\ \hline 
RF2.5 & Il sistema deve permettere di convertire un file .DXF in un file .PNG che riproduce graficamente l'immagine contenuta nel file originale & Obbligatorio &  \textcolor{green!80!blue}{soddisfatto}  \\ \hline 
RF2.6 & Il sistema deve permettere di calcolare la homography matrix utilizzata per la traduzione delle coordinate di tracking relative ad una telecamera & Obbligatorio &  \textcolor{green!80!blue}{soddisfatto}  \\ \hline 
RF2.7 & Il sistema deve permettere di selezionare una telecamera per la modifica & Obbligatorio &  \textcolor{green!80!blue}{soddisfatto}  \\ \hline 
RF3 & Il sistema deve permettere all'utente di generare delle statistiche a partire dai dati di tracking attualmente presenti & Obbligatorio &  \textcolor{orange}{incompleto}  \\ \hline 
RF3.1 & Il sistema deve permettere di trasformare i dati di tracking in coordinate di posizione all'interno di un immagine .PNG che descrive la planimetria del locale, e di salvare tali informazioni in maniera persistente & Obbligatorio &  \textcolor{green!80!blue}{soddisfatto}  \\ \hline 
RF3.2 & Il sistema deve permettere di generare un heatmap grafica che fornisca una rappresentazione grafica dei dati di tracking & Obbligatorio &  \textcolor{green!80!blue}{soddisfatto}  \\ \hline 
RF3.3 & Il sistema deve permettere di generare delle statistiche numeriche a partire dai dati di tracking e di salvarle in maniera persistente & Facoltativo &  \textcolor{red}{non soddisfatto}  \\ \hline 
RF3.3.1 & Il sistema deve permettere di generare la statistica di \textit{Dwell Time} & Facoltativo &  \textcolor{red}{non soddisfatto}  \\ \hline 
RF3.3.2 & Il sistema deve permettere di generare la statistica di \textit{Counting} & Facoltativo &  \textcolor{red}{non soddisfatto}  \\ \hline 
RF3.3.3 & Il sistema deve permettere di generare la statistica di \textit{Waiting Line} & Facoltativo &  \textcolor{red}{non soddisfatto}  \\ \hline 
\end{longtable}
\end{center}

\subsubsection{Requisiti di vincolo} \label{sec:reqvin}
\begin{center}
    \begin{longtable}{ | l | p{5cm} | c | c |}
    \caption{Tabella di soddisfacimento dei requisiti di vincolo} \\
    \hline 
    \textbf{Codice} & \textbf{Descrizione} & \textbf{Tipologia} & \textbf{Esito} \\ \hline
\endfirsthead
\multicolumn{4}{c}% 

{\tablename\ \thetable\ -- \textit{Continued from previous page}} \\
\hline
\textbf{Codice} & \textbf{Descrizione} & \textbf{Tipologia} & \textbf{Esito} \\
\hline
\endhead
\hline \multicolumn{4}{r}{\textit{Continued on next page}} \\
\endfoot
\hline
\endlastfoot
RV1 & Il sistema dev'essere strutturato secondo un'architettura 3-tier & Obbligatorio &  \textcolor{green!80!blue}{soddisfatto}  \\ \hline 
RV2 & Il sistema deve funzionare in ambiente Linux & Obbligatorio &  \textcolor{green!80!blue}{soddisfatto}  \\ \hline 
RV3 & Il sistema deve funzionare in ambiente Windows & Facoltativo &  \textcolor{red}{non soddisfatto}  \\ \hline 
RV4 & Il sistema deve funzionare in ambiente Mac OS X & Desiderabile &  \textcolor{green!80!blue}{soddisfatto}  \\ \hline 
RV5 & Il sistema utilizzerà come strumento di build CMake, verranno quindi forniti i relativi file & Obbligatorio &  \textcolor{green!80!blue}{soddisfatto}  \\ \hline 
\end{longtable}
\end{center}

\subsubsection{Requisiti di qualita} \label{sec:reqqua}
\begin{center}
    \begin{longtable}{ | l | p{5cm} | c | c |}
    \caption{Tabella di soddisfacimento dei requisiti di qualita} \\
    \hline 
    \textbf{Codice} & \textbf{Descrizione} & \textbf{Tipologia} & \textbf{Esito} \\ \hline
\endfirsthead
\multicolumn{4}{c}% 

{\tablename\ \thetable\ -- \textit{Continued from previous page}} \\
\hline
\textbf{Codice} & \textbf{Descrizione} & \textbf{Tipologia} & \textbf{Esito} \\
\hline
\endhead
\hline \multicolumn{4}{r}{\textit{Continued on next page}} \\
\endfoot
\hline
\endlastfoot
RQ1 & Deve essere prodotta documentazione del codice sorgente del software & Obbligatorio &  \textcolor{green!80!blue}{soddisfatto}  \\ \hline 
RQ2 & Il sistema deve garantire la massima indipendenza tra le funzionalità & Desiderabile &  \textcolor{green!80!blue}{soddisfatto}  \\ \hline 
\end{longtable}
\end{center}

